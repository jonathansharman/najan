\section{Numbers} \label{sec:numbers}

Numerals always function grammatically as nouns and signify rational numbers.

\paragraph{Ordinals} Ordinality (\trans{ray}) in Najan is zero-based (unlike in
English, where ordinals are one-based).

\example{Ordinal number}{
	the first person \\
	$\Downarrow$ \\
	the person with ordinality equal to zero
}{
	\trans{i nuh thik anh ray mihl}
}

\todo{ This section below this line may be out of date. Should have just twelve
	easily distinguishable digits, combined using modifiers. }

Najan uses a dozenal positional numeral system. Digits are grouped into threes,
with 1,728 possible values per digit group. Placeholder zeros are not
pronounced, and the eleven non-zero digits have unique names depending on their
place within a digit group. Table~\ref{tab:digit-names} lists the names of
digits in each position.

The preposition \trans{nuh dhihn} (``with negation'') can be used to negate a
numeral. Integers with absolute value less than decimal $12^3$ can be expressed
using just a single digit group. For numbers with magnitude greater or equal to
$12^3$ or with a fractional component, each group of three digits is preceded by
the word \trans{lihnh} and a multiplier. The value of each digit group is its
base value times $12^3$ raised to the power of its multiplier. Note that
negative multipliers express fractional values.

The following extended Backus-Naur form (EBNF) grammar defines the structure of
Najan numerals.

\begin{ebnf}
	\nt{numeral} \is \opt{\t{\naj{dhihn}}} (\nt{group} \alt \plus{\t{\naj{lihnh}} \nt{numeral} \nt{group}})

	\nt{group} \is \nt{triple} \opt{\nt{double}} \opt{\nt{single}}
	\altLine \nt{double} \opt{\nt{single}}
	\altLine \nt{single}
	\altLine \t{\naj{anh}}

	\nt{single} \is \t{\naj{aj}} \alt \t{\naj{az}} \alt \t{\naj{ac}} \alt \t{\naj{ak}} \alt \t{\naj{af}} \alt \t{\naj{ash}} \alt \t{\naj{av}}

	\nt{double} \is \t{\naj{ej}} \alt \t{\naj{ez}} \alt \t{\naj{ec}} \alt \t{\naj{ek}} \alt \t{\naj{ef}} \alt \t{\naj{esh}} \alt \t{\naj{ev}}

	\nt{triple} \is \t{\naj{oj}} \alt \t{\naj{oz}} \alt \t{\naj{oc}} \alt \t{\naj{ok}} \alt \t{\naj{of}} \alt \t{\naj{osh}} \alt \t{\naj{ov}}
\end{ebnf}

\begin{table}
	\caption{Digit names}
	\centering
	\begin{tabular}{llll}
		\toprule
		\multirow{2}{*}{Digit} & \multicolumn{3}{c}{Multiplier}                             \\
		\cmidrule{2-4}
		                       & $\times 1$                     & $\times 8$  & $\times 64$ \\
		\midrule
		0                      & \trans{anh}                    & --          & --          \\
		1                      & \trans{aj}                     & \trans{ej}  & \trans{oj}  \\
		2                      & \trans{az}                     & \trans{ez}  & \trans{oz}  \\
		3                      & \trans{ac}                     & \trans{ec}  & \trans{oc}  \\
		4                      & \trans{ak}                     & \trans{ek}  & \trans{ok}  \\
		5                      & \trans{af}                     & \trans{ef}  & \trans{of}  \\
		6                      & \trans{ash}                    & \trans{esh} & \trans{osh} \\
		7                      & \trans{av}                     & \trans{ev}  & \trans{ov}  \\
		\bottomrule
	\end{tabular}
	\label{tab:digit-names}
\end{table}
