\section{Verbs} \label{sec:verbs}

\subsection{Mood} \label{sec:mood}

Clauses default to the indicative mood but can be changed using a particle
before the clause.

\paragraph{Indicative} Signifies that the speaker believes the clause is true.

\paragraph{Subjunctive} Used for talking about clauses in the abstract,
subjunctive clauses don't signify anything as independent clauses; they are
always used as a dependent clause (e.g. in a hypothetical) or other noun phrase,
serving a role similar to verbal nouns in English. The shortest possible
subjunctive clauses - just the particle and a verb, with no modifiers - can
function exactly like an infinite or gerund in English.

\paragraph{Deontic} Expresses wishes and commands. Najan does not have a
separate imperative mood to express requests or orders; a deontic clause with a
(perhaps implicit) second-person agent serves this purpose.

\paragraph{Polar Interrogative} Express yes/no questions. The particle
\trans{kya} could be translated as ``it is being asked whether [clause]''. The
expectation is that the listener will answer by affirming the question
(\trans{azh}), negating it (\trans{esh}), or objecting to its underlying
premises or implications (\trans{ith}).

\todo{Update this example.}

\example{Polar interrogative sentence}{
	\begin{tabular}{rl}
		\textbf{Alice} & Have you stopped beating your spouse? \\
		\textbf{Bob}   & To the contrary, I never beat her.    \\
	\end{tabular}
}{
	\begin{tabular}{rl}
		% Duplicating text here instead of using a macro to make the
		% table rows and columns work.
		\textbf{Alice} & \naj{kya kxesh goruhp to i zho nih enihd to} \\
		               & ⟨kya kxesh goruhp to i zho nih enihd to      \\
		\textbf{Bob}   & \naj{ith. nu goruhp ko zho}                  \\
		               & ⟨ith. nu goruhp ko zho⟩                      \\
	\end{tabular}
}

Multiple-choice questions can be constructed using the polar interrogative mood
and a logical disjunction, which due to the rewriting rules defined in
\S~\ref{sec:conj-rewriting} has the same semantics as asking multiple yes/no
questions.

\paragraph{Open Interrogative} Used for open-ended questions. An open
interrogative sentence must include at least one occurrence of the interrogative
pronoun, \trans{vo}. The utterance of an open interrogative clause invites the
listener to supply an answer for each occurrence of \trans{vo}. As with polar
questions, \trans{ihth} can be an appropriate response if the question is based
on false premises and therefore cannot be meaningfully answered.

\example{Open interrogative sentence}
{What do you want?}
{\transLine{kwa ya kih FOR vo vihm to}}

\example{Open interrogative sentence with discrete choices}
{Do you want soup or salad?}
{\transLine{kwa ya vihm nhinh zaw thloth pewsh vo to}}

\paragraph{Confirmative Interrogative} This mood is similar to the polar
interrogative except that the speaker already believes the proposition to be
true, at least tentatively. Use of this mood presents the addressee a chance to
contradict the claim if it is false. It may also be used rhetorically when the
speaker is already certain the claim is true. Sentences of this mood are like
tag questions in English (``right?'') and other languages. \trans{odh} and
\trans{ihth} are typical responses to confirmative questions.

\todo{Make these particles more dissimilar and interesting.}

\begin{table}
	\caption{Mood particles}
	\centering
	\begin{tabular}{ll}
		\toprule
		Mood                       & Particle     \\
		\midrule
		Indicative                 & --           \\
		Subjunctive                & \trans{zhe}  \\
		Deontic                    & \trans{ksha} \\
		Polar Interrogative        & \trans{kya}  \\
		Open Interrogative         & \trans{kwa}  \\
		Confirmative Interrogative & \trans{kla}  \\
		\bottomrule
	\end{tabular}
	\label{tab:mood-particles}
\end{table}

\subsection{Aspect} \label{sec:aspect}

Verbs default to the continuous aspect but can be changed using a particle
immediately preceding the verb. (Note that the continuous aspect applies to both
dynamic and stative verbs. Najan does not distinguish between progressive and
continuous aspects.)

\begin{table}
	\caption{Aspect particles}
	\centering
	\begin{tabular}{ll}
		\toprule
		Particle     & Aspect
		\midrule
		--           & Continuous \\
		\trans{the}  & Perfective \\
		\trans{fxe}  & Habitual   \\
		\trans{khon} & Gnomic     \\
		\bottomrule
	\end{tabular}
	\label{tab:aspect-particles}
\end{table}

\subsection{Arguments} \label{sec:arguments}

All arguments to the verb (such as the agent or patient) are expressed via
optional prepositional phrases modifying the verb (\S\ref{sec:modifiers}). Since
prepositional phrases can occur in any order, Najan has very free word order.
Prepositions for expressing common arguments are listed in
Table~\ref{tab:argument-prepositions}.

\begin{table}
	\caption{Argument prepositions}
	\centering
	\begin{tabular}{ll}
		\toprule
		Relation          & Modifier      \\
		\midrule
		Attributee        & \trans{nihm}  \\
		Attribute         & \trans{gha}   \\
		Agent             & \trans{shi}   \\
		Involuntary cause & \trans{slo}   \\
		Patient           & \trans{ruh}   \\
		Instrument        & \trans{lathu} \\
		Recipient         & \trans{dhu}   \\
		\bottomrule
	\end{tabular}
	\label{tab:argument-prepositions}
\end{table}

\todo{Examples. "I destroy the sand castle" vs. "The sea destroys the sand castle"}

\paragraph{Dynamic and Stative Clauses} In English, the subject does not vary
when paired with a dynamic vs. a stative verb. However, in Najan, dynamic
clauses typically include either an ``agent'' or ``involuntary cause'', while
stative clauses typically include an ``attributee''.

Note that because all arguments to the verb are indicated using modifiers, it's
possible for a verb to have multiple arguments of the same type - for instance,
multiple agents - without using grouping phrases.

\todo{Format and translate examples below.}

\trans{shi ko shi to <sing>} implies singing together.

\trans{shi ko ruh <ball> ruh <glove> <throw>} implies I threw the ball
and glove as part of the same action.

\subsection{Tense} \label{sec:tense}

Like arguments to the verb, verb tense is expressed by modifying the verb with a
prepositional phrase. If no such modifiers are present, then the time frame must
be inferred from context. Table~\ref{tab:temporal-modifiers} lists all the
modifiers that affect the tense of a verb phrase, and
Table~\ref{tab:temporal-terms} provides a (nonexhaustive) list of terms commonly
used with temporal modifiers.

\begin{table}
	\caption{Temporal modifiers}
	\centering
	\begin{tabular}{ll}
		\toprule
		Modifier     & Meaning                     \\
		\midrule
		\trans{ve}   & at/on [time point]          \\
		\trans{le}   & for [duration]              \\
		\trans{de}   & during/in [time period]     \\
		\trans{the}  & before [time point]         \\
		\trans{thih} & shortly before [time point] \\
		\trans{thu}  & long before [time point]    \\
		\trans{she}  & after [time point]          \\
		\trans{shih} & shortly after [time point]  \\
		\trans{shu}  & long after [time point]     \\
		\trans{nhe}  & from [time point]           \\
		\trans{ne}   & to [time point]             \\
		             & until [event]               \\
		             & while [event]               \\
		\bottomrule
	\end{tabular}
	\label{tab:temporal-modifiers}
\end{table}

\todo{That looks like inflection to me.}

\begin{table}
	\caption{Common temporal terms}
	\centering
	\begin{tabular}{ll}
		\toprule
		Term            & Meaning                   \\
		\midrule
		\trans{va}      & the present/now           \\
		\trans{tha}     & the past                  \\
		\trans{sha}     & the future                \\
		\trans{gha}     & moment/instant            \\
		\trans{kha}     & time span                 \\
		\trans{thenish} & eternity/forever/all time \\
		\bottomrule
	\end{tabular}
	\label{tab:temporal-terms}
\end{table}

\subsection{Interjections} \label{sec:interjections}

Some avalent verbs function as interjections. Four common interjections are
\trans{azh} (``yes''), \trans{esh} (``no''), \trans{odh} (``indeed'',
``right''), and \trans{ihth} (``to the contrary'').
