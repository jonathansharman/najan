\section{Clauses} \label{sec:verbs}

Syntactically, Najan has only four verbs (Table~\ref{tab:verbs}), which express
only aspect.

\begin{table}
	\caption{Verbs}
	\centering
	\begin{tabular}{ll}
		\toprule
		\trans{the}  & Perfective             \\
		\trans{zhe}  & Continuous/progressive \\
		\trans{fxe}  & Habitual               \\
		\trans{khon} & Gnomic                 \\
		\bottomrule
	\end{tabular}
	\label{tab:verbs}
\end{table}

Modifiers (\S\ref{sec:modifiers}) to these aspectual verbs convey almost all the
semantic information in a clause, including the agent, patient, and even the
action or state of being itself. Since such modifiers can occur in any order,
Najan has very free word order.

The modifier \trans{nuh} (``with/having [a property]'') is used along with an
appropriate noun phrase to convey the action (for dynamic clauses) or state of
being (for stative clauses) associated with the verb. In fact, \trans{nuh} can
be used to convey any kind of argument to the verb. For instance, to express a
first-person agent, one could use a phrase that translates literally to ``with
agent of I''. However, certain thematic relations have dedicated modifiers,
which can express such arguments more succinctly and are listed in
Table~\ref{tab:thematic-relations}.

\begin{table}
	\caption{Thematic relational modifiers}
	\centering
	\begin{tabular}{ll}
		\toprule
		Relation                   & Modifier     \\
		\midrule
		Agent                      & \trans{shi}  \\
		Involuntary cause          & \trans{otq}  \\
		Attributee [of a property] & \trans{nihm} \\
		Patient                    & \trans{ruh}  \\
		Instrument                 & \trans{gha}  \\
		Recipient                  & \trans{dhu}  \\
		\bottomrule
	\end{tabular}
	\label{tab:thematic-relations}
\end{table}

\todo{Examples. "I destroy the sand castle" vs. "The sea destroys the sand castle"}

\paragraph{Dynamic and Stative Clauses} In English, the subject does not vary
when paired with a dynamic vs. a stative verb. However, in Najan, dynamic
clauses typically include either an ``agent'' or ``involuntary cause'', while
stative clauses typically include an ``attributee''.

Note that because all arguments to the verb are indicated using modifiers, it's
possible for a verb to have multiple actions, agents, etc. associated with a
single verb, without using grouping phrases.

\todo{Format and translate examples below.}

``I threw and broke your toy.'' Since both ``throw'' and ``break'' modify
\trans{the} in this case, they become semantically part of a single action.

\trans{shi ko shi to nuh <sing> the} implies singing together.

\trans{shi ko ruh <ball> ruh <glove> nuh <throw> the} implies I threw the ball
and glove as part of the same action.

\todo{Anything below this line in this section may be outdated.}

\subsection{Stative Verbs} \label{ssec:stative-verbs}

\example{Stative sentence using \trans{ya}}{
	I love you.
}{
	\transLine{ya kih FOR to elihf ko} \\
	(literally ``I with love for you exist.'') \\
}

\subsection{Mood} \label{ssec:mood}

Structurally, Najan only has one mood, which is the indicative. The semantics of
other moods can be expressed by modifying an indicative sentence using a
monovalent sentence modifier. Table~\ref{tab:mood-modifiers} lists each mood
with its corresponding modifier.

\paragraph{Indicative} The indicative mood signifies that the speaker believes
the clause is true. An unmodified sentence is indicative.

\paragraph{Deontic} Deontic clauses express wishes and commands. Najan does not
have a separate imperative mood to express requests or orders; a deontic clause
with a second-person subject serves this purpose.

\paragraph{Polar Interrogative} Polar interrogative clauses express yes/no
questions. The modifier \trans{kya} could be translated as ``it is being asked
whether [sentence]''. The expectation is that the listener will answer by
affirming the question (\trans{azh}), negating it (\trans{esh}), or objecting to
its underlying premises or implications (\trans{ith}).

\example{Polar interrogative sentence}{
	\begin{tabular}{rl}
		\textbf{Alice} & Have you stopped beating your spouse? \\
		\textbf{Bob}   & To the contrary, I never beat her.    \\
	\end{tabular}
}{
	\begin{tabular}{rl}
		% Duplicating text here instead of using a macro to make the
		% table rows and columns work.
		\textbf{Alice} & \naj{kya kxesh goruhp to i zho nih enihd to} \\
		               & ⟨kya kxesh goruhp to i zho nih enihd to      \\
		\textbf{Bob}   & \naj{ith. nu goruhp ko zho}                  \\
		               & ⟨ith. nu goruhp ko zho⟩                      \\
	\end{tabular}
}

\paragraph{Open Interrogative} Used for open-ended questions. An open
interrogative sentence must include at least one occurrence of the interrogative
pronoun, \trans{vo}. The utterance of an open interrogative clause invites the
listener to supply an answer for each occurrence of \trans{vo}. As with polar
questions, \trans{ihth} can be an appropriate response if the question is based
on false premises and therefore cannot be meaningfully answered.

\example{Open interrogative sentence}
{What do you want?}
{\transLine{kwa ya kih FOR vo vihm to}}

\example{Open interrogative sentence with discrete choices}
{Do you want soup or salad?}
{\transLine{kwa ya vihm nhinh zaw thloth pewsh vo to}}

\paragraph{Confirmative Interrogative} This mood is similar to the polar
interrogative except that the speaker already believes the proposition to be
true, at least tentatively. Use of this mood presents the addressee a chance to
contradict the claim if it is false. It may also be used rhetorically when the
speaker is already certain the claim is true. Sentences of this mood are like
tag questions in English and other languages. \trans{odh} and \trans{ihth} are
typical responses to confirmative questions.

\begin{table}
	\caption{Mood modifiers}
	\centering
	\begin{tabular}{ll}
		\toprule
		Mood                       & Modifier     \\
		\midrule
		Indicative                 & --           \\
		Deontic                    & \trans{ksha} \\
		Polar Interrogative        & \trans{kya}  \\
		Open Interrogative         & \trans{kwa}  \\
		Confirmative Interrogative & \trans{kla}  \\
		\bottomrule
	\end{tabular}
	\label{tab:mood-modifiers}
\end{table}

\subsection{Tense} \label{ssec:tense}

Temporal predicate modifiers express tense. If no such modifiers are present,
then the time frame must be inferred from context.
Table~\ref{tab:temporal-modifiers} lists all the modifiers that affect the tense
of a verb phrase, and Table~\ref{tab:temporal-terms} provides a (nonexhaustive)
list of terms commonly used with temporal modifiers.

\begin{table}
	\caption{Temporal modifiers}
	\centering
	\begin{tabular}{ll}
		\toprule
		Modifier     & Meaning                     \\
		\midrule
		\trans{ve}   & at/on [time point]          \\
		\trans{le}   & for [duration]              \\
		\trans{de}   & during/in [time period]     \\
		\trans{the}  & before [time point]         \\
		\trans{thih} & shortly before [time point] \\
		\trans{thu}  & long before [time point]    \\
		\trans{she}  & after [time point]          \\
		\trans{shih} & shortly after [time point]  \\
		\trans{shu}  & long after [time point]     \\
		\trans{nhe}  & from [time point]           \\
		\trans{ne}   & to [time point]             \\
		             & until [event]               \\
		             & while [event]               \\
		\bottomrule
	\end{tabular}
	\label{tab:temporal-modifiers}
\end{table}

\todo{That looks like inflection to me.}

\begin{table}
	\caption{Common temporal terms}
	\centering
	\begin{tabular}{ll}
		\toprule
		Term            & Meaning                   \\
		\midrule
		\trans{va}      & the present/now           \\
		\trans{tha}     & the past                  \\
		\trans{sha}     & the future                \\
		\trans{gha}     & moment/instant            \\
		\trans{kha}     & time span                 \\
		\trans{thenish} & eternity/forever/all time \\
		\bottomrule
	\end{tabular}
	\label{tab:temporal-terms}
\end{table}
