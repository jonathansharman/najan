\section{Orthography} \label{sec:orthography}

Najan orthography is perfectly phonemic, i.e. there is a bijection between
graphemes and phonemes. Table~\ref{tab:alphabet} lists the thirty-seven letters
of the Najan alphabet, along with the transliteration and IPA transcription for
each letter.

Sentences are written left-to-right, then top-to-bottom. The only punctuation
mark in written Najan is the sentence separator, ``\naj{.}'', which is used to
separate independent sentences and/or interjections.

\begin{table}
	\caption{The Najan alphabet}
	\centering
	\resizebox{\columnwidth}{!}{\begin{tabular}{cccccc}
			\toprule
			\grapheme{k}  & \grapheme{g}  & \grapheme{t}  & \grapheme{d}  & \grapheme{p} & \grapheme{b} \\
			\transIPA{k}  & \transIPA{g}  & \transIPA{t}  & \transIPA{d}  & \transIPA{p} & \transIPA{b} \\
			\midrule

			\grapheme{kq} & \grapheme{gq} & \grapheme{tq} & \grapheme{dq} & \grapheme{c} & \grapheme{j} \\
			\transIPA{kq} & \transIPA{gq} & \transIPA{tq} & \transIPA{dq} & \transIPA{c} & \transIPA{j} \\
			\midrule

			\grapheme{kh} & \grapheme{gh} & \grapheme{x}  & \grapheme{r}  & \grapheme{s} & \grapheme{z} \\
			\transIPA{kh} & \transIPA{gh} & \transIPA{x}  & \transIPA{r}  & \transIPA{s} & \transIPA{z} \\
			\midrule

			\grapheme{sh} & \grapheme{zh} & \grapheme{th} & \grapheme{dh} & \grapheme{f} & \grapheme{v} \\
			\transIPA{sh} & \transIPA{zh} & \transIPA{th} & \transIPA{dh} & \transIPA{f} & \transIPA{v} \\
			\midrule

			\grapheme{nh} & \grapheme{n}  & \grapheme{m}  & \grapheme{y}  & \grapheme{l} & \grapheme{w} \\
			\transIPA{nh} & \transIPA{n}  & \transIPA{m}  & \transIPA{y}  & \transIPA{l} & \transIPA{w} \\
			\midrule
		\end{tabular}}
	\resizebox{\columnwidth}{!}{\begin{tabular}{ccccccc}
			\grapheme{uh} & \grapheme{a} & \grapheme{e} & \grapheme{ih} & \grapheme{i} & \grapheme{u} & \grapheme{o} \\
			\transIPA{uh} & \transIPA{a} & \transIPA{e} & \transIPA{ih} & \transIPA{i} & \transIPA{u} & \transIPA{o} \\
			\bottomrule
		\end{tabular}}
	\label{tab:alphabet}
\end{table}

\subsection{Numbers}

\todo{Switch to dozenal.}

Najan uses an octal positional numeral system and thus has eight numeric glyphs
(0-7): \naj{0}, \naj{1}, \naj{2}, \naj{3}, \naj{4}, \naj{5}, \naj{6}, and
\naj{7}. Numerals are typically written using these glyphs (from most to least
significant digit) rather than spelling them out as they are pronounced (as
described in \S\ref{sec:morphology}).
