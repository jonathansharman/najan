\section{Syntax} \label{sec:syntax}

The formal grammar of the language is defined by the following extended
Backus-Naur form (EBNF) grammar. This is a deterministic context-free grammar,
making it amenable to automatic parsing.

\begin{ebnf}
	\nt{sentence} \is \opt{\nt{mood particle}} \nt{clause}
	\\
	\nt{clause} \is \nt{predicate}
	\altLine \t{\trans{nunh}} \nt{clause}
	\altLine \nt{binary clause conj.} \nt{clause} \nt{clause}
	\\
	\nt{predicate} \is \nt{verb}
	\altLine \nt{predicate mod.} \nt{NP} \nt{predicate}
	\altLine \nt{predicate conj.} \nt{predicate} \nt{predicate}
	\\
	\nt{noun phrase (NP)} \is \nt{quant. NP}
	\altLine \nt{unquant. NP}
	\altLine \nt{clause}
	\altLine \nt{quotation}
	\altLine \nt{conj.} \nt{NP} \nt{NP}
	\\
	\nt{quant. NP} \is \nt{det.} \plus{\nt{pronoun}} \nt{unquant. NP}
	\\
	\nt{unquant. NP} \is \nt{noun} \alt \nt{pronoun}
	\altLine \nt{prep.} \nt{NP} \nt{unquant. NP}
	\\
	\nt{quotation} \is \t{\trans{ca}} \star{\nt{quoted word}} \t{\trans{ca}}
	\\
	\nt{quoted word} \is \nt{word} - \t{\trans{ca}} - \t{\trans{nosh}} - \t{\trans{tqa}}
	\altLine \t{\trans{tqa}} \t{\trans{ca}}
	\altLine \t{\trans{tqa}} \t{\trans{nosh}}
	\altLine \t{\trans{tqa}} \t{\trans{tqa}}
\end{ebnf}

\todo{Example}

\paragraph{Quotation} A quotation begins and ends with the particle \trans{ca},
acting as a noun signifying the literal contained words themselves. The escape
particle \trans{tqa} is used within a quotation just before \trans{ca},
\trans{nosh}, or \trans{tqa} to indicate that the following particle should be
interpreted as part of the quotation, not as a particle within the sentence
containing the quotation.

\example{Nested quotation using particles}{
	You say ``I say `no'''.
}{
	\transLine{ka ca ka ca esh tqa sa tihz ko sa tihz to}
}
