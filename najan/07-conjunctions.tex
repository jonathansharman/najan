\section{Conjunctions} \label{sec:conjunctions}

In Najan, conjunctions are restricted to logical connectives, i.e. functions of
Boolean values. Many words considered conjunctions in other languages, such as
English ``because'', are instead classified as modifiers in Najan. Conjunctions
can connect sentences, predicates, or terms and can be unary, binary, or
variadic. In the case of variadic conjunctions, \trans{sa} marks the end of the
list of arguments. Table~\ref{tab:conjunctions} lists all conjunctions in Najan.

\begin{table}
	\caption{Conjunctions}
	\centering
	\begin{tabular*}{\columnwidth}{@{\extracolsep{\fill}} llll}
		\toprule
		Function & Sentence & Predicate & Term \\
		\midrule
		$\lnot$ ({\sc not}) & \trans{nunh} & -- & -- \\
		$\land$ ({\sc and}) & \trans{senh} & \trans{sen} & \trans{sem} \\
		$\lor$ ({\sc or}) & \trans{shenh} & \trans{shen} & \trans{shem} \\
		$\oplus$ ({\sc xor}) & \trans{cenh} & \trans{cen} & \trans{cem} \\
		$\rightarrow$ ({\sc implies}) & \trans{tqanh} & \trans{tqan} & \trans{tqam} \\
		\bottomrule
	\end{tabular*}
	\label{tab:conjunctions}
\end{table}

\todo{ Do I need three versions of each conjunction, or can some of them be
	shared and distinguished by syntax? }

Note that biconditional statements (``if and only if'') are equivalent to
	{\sc xnor}, which can be expressed by combining {\sc not} and {\sc xor} conjunctions.

Note that \trans{tqanh} only denotes material implication; it does not imply a
causal relationship.

Sentential conjunctions are the fundamental and most general form of
conjunction. Semantically, any conjunction of terms or predicates can be
expressed using a sentential conjunction instead, though usually less concisely.
As with the conversion from quantified terms to forward assignments
(\S\ref{sec:binding}), this is a mechanical transformation. A sentence can be
converted to only sentential conjunctions as follows:

\begin{enumerate}
	\item Factor out in-place quantified terms to forward assignments, as
	      described in \S\ref{sec:binding}. Now consider the subsentence after
	      all forward assignments.
	\item Scanning left to right, if there is a term or predicate conjunction,
	      replace the containing sentence with the corresponding sentential
	      conjunction with copies of the sentence, each holding one of the conjunction
	      arguments in place of the original conjunction.
	\item If you performed any substitutions, repeat steps 2-3 for each new
	      subsentence.
\end{enumerate}

\example{Conversion from a term conjunction to a sentence conjunction}{
	I have a cat and a dog.
}{
	\transLine{lut ko sem a zho kanaz a co nhiruh} \\
	$\Downarrow$ \\
	\transLine{a zho kanaz a co nhiruh lut ko sem zho co} \\
	$\Downarrow$ \\
	\transLine{a zho kanaz a co nhiruh senh lut ko zho lut ko co}. \\
}

Special care should be taken when translating conjunctions from English to
Najan. A simple literal translation often provides the wrong semantics. Consider
the following translation attempt:

\example{Incorrect translation of English conjunction}
{I want an apple or an orange.}
{WRONG: \trans{zihm ko cem na zho duhxihn na co rhabuhm}}

The translated sentence actually means ``Either I want an apple, or I want an
orange.'' A more faithful translation of the original sentence would be as
follows:

\example{Correct translation of conjunction using union modifier}
{I want an apple or an orange.}
{\transLine{zihm ko na zho zaw duhxihn rhabuhm}}

This literally means that the speaker wants any single item that is either an
apple or an orange.
