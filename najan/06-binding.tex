\section{Binding and Quantification} \label{sec:binding}

\paragraph{Quantified Terms} A quantified term binds one or more pronouns to a
complement, according to the semantics of a determiner, and signifies the bound
pronouns in-place. Generally, such pronouns remain bound until they are rebound
to a different complement. However, any pronouns bound using the universal
determiner \trans{u}---or in any subsequent quantified term within the same
clause---become unbound at the end of the current clause.

\todo{ Explain the semantics of multiple pronouns. (I think it'll be equivalent
	to repeating the containing phrase.) }

\todo{ Examples illustrating why it works this way. "Every man X has a horse Y;
	Y is a good horse" doesn't make sense. "A man X has a horse Y; Y is a good
	horse" does }

\paragraph{Quotations} Quotations cannot affect the bindings of surrounding
pronouns. Any assignable pronouns contained in a quotation may refer to
completely different antecedents that in the surrounding context or even to
nothing at all, e.g. when using a quotation to talk about the pronouns
themselves.

\subsection{Pronouns} \label{sec:pronouns}

\paragraph{Assignable} Assignable pronouns are those which can be bound
explicitly using determiners. There are four assignable pronouns: \trans{zho},
\trans{co}, \trans{rho}, and \trans{so}. Once all available pronouns have been
assigned, subsequent assignments must overwrite an existing assignment.

\paragraph{Personal} Najan has four personal pronouns, which are listed in
Table~\ref{tab:personal-pronouns}.

\begin{table}
	\caption{Personal pronouns}
	\centering
	\begin{tabular}{lll}
		\toprule
		               & Singular   & Plural       \\
		\midrule
		\nth{1}-person & \trans{ko} & \trans{kxho} \\
		\nth{2}-person & \trans{to} & \trans{txho} \\
		\bottomrule
	\end{tabular}
	\label{tab:personal-pronouns}
\end{table}

Note that the first-person plural (\trans{kxho}) is used only in speech or
writing considered to have multiple speakers or authors, to refer to the group a
spokesperson represents or to the coauthors of a jointly written document. In
other words, the first-person plural in Najan does not merely indicate
association with the speaker; it indicates multiple speakers. Unlike in English,
the first-person plural should not be used to refer to one's group when speaking
as an individual or to refer to oneself and the addressee together.

The choice between second-person singular or plural is more intuitive, from an
English-language perspective. If there is a single primary addressee, the
singular is used. If there are multiple addressees, the plural is used.

\todo{I could probably get rid of the plural personal pronouns.}

\paragraph{Interrogative} The interrogative pronoun \trans{vo} is used in place
of unknown terms. It is also essential to the formation of interrogative
sentences, where it indicates any missing information that the asker wishes to
know.

\subsection{Determiners} \label{sec:determiners}

\paragraph{Universal} The universal determiner \trans{u} is used to express that
its immediately enclosing clause holds for each instance of its complement,
functioning similarly to English ``each'', ``every'', or ``all''. When it binds
$k$ pronouns in a single quantified term, the clause applies to every
$k$-permutation of the elements. As noted above, the universal determiner is
unique in that it only binds pronouns for the duration of the containing clause.

\example{Universal determiner}
{Everyone hugged each other.}
{\transLine{u zho co mil the buda va zho co}}

\paragraph{Existential} The existential determiner \trans{a} binds pronouns to
individual instances of a complement, similar to many uses of English
``a''/``an''. When multiple pronouns are given in a forward assignment, each
signifies a unique instance. Pronouns assigned by two separate existential
determiners applied to the same complement may or may not share a referent.

\example{Existential determiner}
{I have a dog.}
{\transLine{ya lut ko a zho nhiruh}}

\todo{ Rework this old example, talking about shifting modifiers around. Also
	include an example using a specialized ``for'' modifier, for when the
	pronouns need to be defined together but used separately? }

\example{Using forward assignment to change semantics}{
	Every person has an ancestor. \\
	$\Downarrow$ \\
	For each person, there's a person that's an ancestor to the first. \\
}{
	\transLine{u zho mil ya nhinh a co mil OF runat zho} \\
	$\Downarrow$ \\
	\transLine{u zho mil a co mil ya nhinh co OF runat zho} \\
}

\paragraph{Definite} When the set of entities an unquantified term refers to is
a singleton, i.e. when it has only one instance, the definitive determiner
\trans{i} can be used to refer to that single instance, functioning much the
same as the English determiner ``the'' in many cases.

\example{Definite determiner}{
	A dog wagged its tail. \\
	$\Downarrow$ \\
	Before now, a dog wagged the tail belonging to it.
}{
	\transLine{the va threre a zho nhiruh i co sex zho shuht}
}

\todo{Do I really want a definite determiner?}

\paragraph{Demonstrative} Demonstrative determiners function just like the
definite determiner except that the complement need not be a singleton. Instead,
the particular instance of the complement is understood by context, e.g. by
physical gesturing or by common understanding. Najan has proximal
(\trans{ihla}), medial (\trans{aya}), and distal (\trans{owa}) demonstrative
determiners, based on the distance of the referent from the speaker in space and
time. These correspond roughly to English ``this'', ``that'', and ``yonder'',
respectively. The concept of distance with regard to demonstrative determiners
can be more abstract than just spatiotemporal distance to a physical object;
e.g. it could be a measure of the relevance of the complement to the present
conversation.

\paragraph{Indefinite} The indefinite determiner \trans{na} acts like the
English word ``any'', meaning a single instance of the complement where the
particular instance does not matter. As with the existential determiner, it can
bind multiple pronouns at once, in which case each referent is unique. Note that
many English sentences using ``a''/``an'' are best translated using \trans{na}
rather than \trans{a}.

\example{Indefinite determiner}{
	I want a book.
}{
	\transLine{ka na zho suaysh zihm ko}
	\\
	acting-on any pron. book want I
}

\example{Indefinite determiner binding two pronouns}{
	I want two books.
}{
	\transLine{wa na zho co suaysh ka sem zho co zihm ko}
	\\
	forward-assignment any pron.-1 pron.-2 book acting-on and pron.-1 pron.-2 want I
}

\todo{ This would probably be better translated as "I want any books numbering
	two" unless the specific two books are going to be referenced individually
	later. Might want to come up with a better example or add a sentence using
	the books. }

\subsection{Cancellation} \label{sec:cancellation}

The cancellation particle \trans{nosh} can be used to cancel the entire
preceding independent sentence, even if it is only partially uttered. The
conversation then proceeds as if the utterance had never occurred, including
reverting the context to its previous state. Speakers and listeners alike can
use \trans{nosh} at any time in a conversation. Cancellation is useful when the
speaker makes a mistake (like saying ``I mean'') or when the listener does not
understand something (like saying ``What?'').

Note that for the sake of simplicity and clarity, the formal grammar
(\S\ref{sec:syntax}) does not handle cancellation. It is relatively
straightforward for a parser to detect cancellation (e.g., by checking for an
unescaped cancellation particle during tokenization), but expressing it formally
overcomplicates the grammar.
