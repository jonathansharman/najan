\section{Conjunctions} \label{sec:conjunctions}

In Najan, conjunctions are restricted to logical connectives, i.e. functions of
Boolean values. Many words considered conjunctions in other languages, such as
English ``because'', are instead classified as modifiers in Najan. Conjunctions
can connect sentences, predicates, or terms and can be unary, binary, or
variadic. In the case of variadic conjunctions, \trans{sa} marks the end of the
list of arguments. Table~\ref{tab:conjunctions} lists all conjunctions in Najan.

\begin{table}
	\caption{Conjunctions}
	\centering
	\begin{tabular*}{\columnwidth}{@{\extracolsep{\fill}} llll}
		\toprule
		Function & Sentence & Predicate & Term \\
		\midrule
		$\lnot$ ({\sc not}) & \trans{nunh} & -- & -- \\
		$\land$ ({\sc and}) & \trans{senh} & \trans{sen} & \trans{sem} \\
		$\lor$ ({\sc or}) & \trans{shenh} & \trans{shen} & \trans{shem} \\
		$\oplus$ ({\sc xor}) & \trans{cenh} & \trans{cen} & \trans{cem} \\
		$\rightarrow$ ({\sc implies}) & \trans{tqanh} & \trans{tqan} & \trans{tqam} \\
		\bottomrule
	\end{tabular*}
	\label{tab:conjunctions}
\end{table}

\todo{ Do I need three versions of each conjunction, or can some of them be
	shared and distinguished by syntax? }

Note that biconditional statements (``if and only if'') are equivalent to
	{\sc xnor}, which can be expressed by combining {\sc not} and {\sc xor} conjunctions.

Note that \trans{tqanh} only denotes material implication; it does not imply a
causal relationship.

\subsection{Conjunction Rewriting Rules} \label{sec:conj-rewriting}

Sentential conjunctions are the fundamental and most general form of
conjunction. Semantically, any conjunction of terms or predicates can be
expressed using a sentential conjunction instead, though usually less concisely.
As with the conversion from quantified terms to forward assignments
(\S\ref{sec:binding}), this is a mechanical transformation. A sentence can be
converted to only sentential conjunctions as follows:

\begin{enumerate}
	\item Factor out in-place quantified terms to forward assignments, as
	      described in \S\ref{sec:binding}. Now consider the subsentence after
	      all forward assignments.
	\item Scanning left to right, if there is a term or predicate conjunction,
	      replace the containing sentence with the corresponding sentential
	      conjunction with copies of the sentence, each holding one of the conjunction
	      arguments in place of the original conjunction.
	\item If you performed any substitutions, repeat steps 2-3 for each new
	      subsentence.
\end{enumerate}

\example{Conversion from a term conjunction to a sentence conjunction}{
	I have a cat and a dog.
}{
	\transLine{lut ko sem a zho kanaz a co nhiruh} \\
	$\Downarrow$ \\
	\transLine{a zho kanaz a co nhiruh lut ko sem zho co} \\
	$\Downarrow$ \\
	\transLine{a zho kanaz a co nhiruh senh lut ko zho lut ko co}. \\
}

Special care should be taken when translating conjunctions from English to
Najan. A simple literal translation often provides the wrong semantics. Consider
the following translation attempt:

\example{Incorrect translation of English conjunction}
{I want an apple or an orange.}
{WRONG: \trans{zihm ko cem na zho duhxihn na co rhabuhm}}

The translated sentence actually means ``Either I want an apple, or I want an
orange.'' A more faithful translation of the original sentence would be as
follows:

\example{Correct translation of conjunction using union modifier}
{I want an apple or an orange.}
{\transLine{zihm ko na zho zaw duhxihn rhabuhm}}

This literally means that the speaker wants any single item that is either an
apple or an orange.

\section{Verbs} \label{sec:verbs}

\subsection{Mood} \label{sec:mood}

Clauses default to the subjunctive mood but can be changed using a particle
before the clause.

\paragraph{Subjunctive} Used for talking about clauses in the abstract,
subjunctive clauses don't signify anything as independent clauses; they are
always used as a dependent clause (e.g. in a hypothetical) or other noun phrase,
serving a role similar to verbal nouns in English. The shortest possible
subjunctive clauses - just the particle and a verb, with no modifiers - can
function exactly like an infinitive or gerund in English.

\paragraph{Indicative} Signifies that the speaker believes the clause is true.

\paragraph{Deontic} Expresses wishes and commands. Najan does not have a
separate imperative mood to express requests or orders; a deontic clause with a
(perhaps implicit) second-person agent serves this purpose.

\paragraph{Polar Interrogative} Express yes-no questions. The particle
\trans{kya} could be translated as ``it is being asked whether [clause]''. The
expectation is that the listener will answer by affirming the question
(\trans{azh}), negating it (\trans{esh}), or objecting to its underlying
premises or implications (\trans{ith}).

\todo{Update this example.}

\example{Polar interrogative sentence}{
	\begin{tabular}{rl}
		\textbf{Alice} & Have you stopped beating your spouse? \\
		\textbf{Bob}   & To the contrary, I never beat her.    \\
	\end{tabular}
}{
	\begin{tabular}{rl}
		% Duplicating text here instead of using a macro to make the
		% table rows and columns work.
		\textbf{Alice} & \naj{kya kxesh goruhp to i zho nih enihd to} \\
		               & ⟨kya kxesh goruhp to i zho nih enihd to      \\
		\textbf{Bob}   & \naj{ith. nu goruhp ko zho}                  \\
		               & ⟨ith. nu goruhp ko zho⟩                      \\
	\end{tabular}
}

Multiple-choice questions can be constructed using the polar interrogative mood
and a logical disjunction, which due to the rewriting rules defined in
\S~\ref{sec:conj-rewriting} has the same semantics as asking multiple yes/no
questions.

\paragraph{Open Interrogative} Used for open-ended questions. An open
interrogative sentence must include at least one occurrence of the interrogative
pronoun, \trans{vo}. The utterance of an open interrogative clause invites the
listener to supply an answer for each occurrence of \trans{vo}. As with polar
questions, \trans{ihth} can be an appropriate response if the question is based
on false premises and therefore cannot be meaningfully answered.

\example{Open interrogative sentence}
{What do you want?}
{\transLine{kwa ya kih FOR vo vihm to}}

\example{Open interrogative sentence with discrete choices}
{Do you want soup or salad?}
{\transLine{kwa ya vihm nhinh zaw thloth pewsh vo to}}

\paragraph{Confirmative Interrogative} This mood is similar to the polar
interrogative except that the speaker already believes the proposition to be
true, at least tentatively. Use of this mood presents the addressee a chance to
contradict the claim if it is false. It may also be used rhetorically when the
speaker is already certain the claim is true. Sentences of this mood are like
tag questions in English (``right?'') and other languages. \trans{odh} and
\trans{ihth} are typical responses to confirmative questions.

\todo{Make these particles more dissimilar and interesting.}

\begin{table}
	\caption{Mood particles}
	\centering
	\begin{tabular}{ll}
		\toprule
		Mood                       & Particle     \\
		\midrule
		Subjunctive                & --           \\
		Indicative                 & \trans{zhe}  \\
		Deontic                    & \trans{ksha} \\
		Polar Interrogative        & \trans{kya}  \\
		Open Interrogative         & \trans{kwa}  \\
		Confirmative Interrogative & \trans{kla}  \\
		\bottomrule
	\end{tabular}
	\label{tab:mood-particles}
\end{table}

\subsection{Aspect} \label{sec:aspect}

Verbs default to the continuous aspect but can be changed using a particle
immediately preceding the verb. (Note that the continuous aspect applies to both
dynamic and stative verbs. Najan does not distinguish between progressive and
continuous aspects.)

\begin{table}
	\caption{Aspect particles}
	\centering
	\begin{tabular}{ll}
		\toprule
		Particle     & Aspect
		\midrule
		--           & Continuous \\
		\trans{the}  & Perfective \\
		\trans{fxe}  & Habitual   \\
		\trans{khon} & Gnomic     \\
		\bottomrule
	\end{tabular}
	\label{tab:aspect-particles}
\end{table}

\subsection{Tense} \label{sec:tense}

Like arguments to the verb, verb tense is expressed by modifying the verb with a
prepositional phrase. If no such modifiers are present, then the time frame must
be inferred from context. Table~\ref{tab:temporal-modifiers} lists all the
modifiers that affect the tense of a verb phrase, and
Table~\ref{tab:temporal-terms} provides a (nonexhaustive) list of terms commonly
used with temporal modifiers.

\begin{table}
	\caption{Temporal modifiers}
	\centering
	\begin{tabular}{ll}
		\toprule
		Modifier     & Meaning                     \\
		\midrule
		\trans{ve}   & at/on [time point]          \\
		\trans{le}   & for [duration]              \\
		\trans{de}   & during/in [time period]     \\
		\trans{the}  & before [time point]         \\
		\trans{thih} & shortly before [time point] \\
		\trans{thu}  & long before [time point]    \\
		\trans{she}  & after [time point]          \\
		\trans{shih} & shortly after [time point]  \\
		\trans{shu}  & long after [time point]     \\
		\trans{nhe}  & from [time point]           \\
		\trans{ne}   & to [time point]             \\
		             & until [event]               \\
		             & while [event]               \\
		\bottomrule
	\end{tabular}
	\label{tab:temporal-modifiers}
\end{table}

\todo{That looks like inflection to me.}

\begin{table}
	\caption{Common temporal terms}
	\centering
	\begin{tabular}{ll}
		\toprule
		Term            & Meaning                   \\
		\midrule
		\trans{va}      & the present/now           \\
		\trans{tha}     & the past                  \\
		\trans{sha}     & the future                \\
		\trans{gha}     & moment/instant            \\
		\trans{kha}     & time span                 \\
		\trans{thenish} & eternity/forever/all time \\
		\bottomrule
	\end{tabular}
	\label{tab:temporal-terms}
\end{table}

\subsection{Interjections} \label{sec:interjections}

Some avalent verbs function as interjections. Four common interjections are
\trans{azh} (``yes''), \trans{esh} (``no''), \trans{odh} (``indeed'',
``right''), and \trans{ihth} (``to the contrary'').

\section{Modifiers} \label{sec:modifiers}

\todo{ This entire section below this line may be outdated and in need of
	reworking/redistribution. }

Modifiers change the meaning of sentences, predicates, or terms. They serve the
roles that prepositions, adjectives, adverbs, and certain coordinating
conjunctions serve in English. A modifier precedes any additional arguments
(which are terms), depending on the valency of the modifier, which precede the
modified object.

\subsection{Sentence Modifiers}

\begin{table}
	\caption{Sentence modifiers}
	\centering
	\begin{tabular}{ll}
		\toprule
		Modifier     & Meaning                                  \\
		\midrule
		\trans{bihn} & because [reason]                         \\
		\trans{vihr} & so/thus/therefore [consequence]          \\
		             & with respect to/regarding/concerning ... \\
		\bottomrule
	\end{tabular}
	\label{tab:sentence-modifiers}
\end{table}

\subsection{Predicate Modifiers}

\begin{table}
	\caption{Spatial modifiers}
	\centering
	\begin{tabular}{ll}
		\toprule
		Modifier & Meaning                        \\
		\midrule
		\naj{}   & at [position]                  \\
		\naj{}   & away from [position/direction] \\
		\naj{}   & to [position/direction]        \\
		\naj{}   & towards [position/direction]   \\

		\bottomrule
	\end{tabular}
	\label{tab:spatial-modifiers}
\end{table}

\subsection{Term Modifiers}

\begin{table}
	\caption{Abstract modifiers}
	\centering
	\begin{tabular}{ll}
		\toprule
		Modifier    & Meaning                    \\
		\midrule
		\trans{nih} & of/related to [term]       \\
		\trans{sex} & component/member of [term] \\
		\trans{sih} & such that [sentence]       \\
		\trans{kih} & with/having [property]     \\
		\bottomrule
	\end{tabular}
	\label{tab:abstract-modifiers}
\end{table}

\begin{table}
	\caption{Mathematical modifiers}
	\centering
	\begin{tabular}{ll}
		\toprule
		Modifier      & Meaning                                    \\
		\midrule
		\trans{nhinh} & belonging to/element of [set]              \\
		\trans{vekh}  & subset of [set]                            \\
		\trans{vekq}  & proper subset of [set]                     \\
		\trans{ghaf}  & superset of [set]                          \\
		\trans{gqaf}  & proper superset of [set]                   \\
		\trans{zaw}   & union with [set]                           \\
		\trans{tew}   & intersection with [set]                    \\
		\trans{khep}  & except/set-minus [set]                     \\
		\trans{shi}   & equivalent to/identical to [term]          \\
		\trans{thik}  & equal to [term]                            \\
		\trans{muhl}  & not equal to ...                           \\
		\trans{thuhm} & less than or equal to ...                  \\
		\trans{thuhl} & less than ...                              \\
		\trans{shuhm} & greater than or equal to ...               \\
		\trans{shuhl} & greater than ...                           \\
		\trans{ray}   & preceded by [cardinal number] predecessors \\
		\bottomrule
	\end{tabular}
	\label{tab:mathematical-modifiers}
\end{table}

\todo{Lots of inflection here as well.}

\section{Numbers} \label{sec:numbers}

Numerals always function grammatically as nouns and signify rational numbers.

\paragraph{Ordinals} Ordinality (\trans{ray}) in Najan is zero-based (unlike in
English, where ordinals are one-based).

\example{Ordinal number}{
	the first person \\
	$\Downarrow$ \\
	the person with ordinality equal to zero
}{
	\trans{i nuh thik anh ray mihl}
}

\todo{ This section below this line may be out of date. Should have just twelve
	easily distinguishable digits, combined using modifiers. }

Najan uses a dozenal positional numeral system. Digits are grouped into threes,
with 1,728 possible values per digit group. Placeholder zeros are not
pronounced, and the eleven non-zero digits have unique names depending on their
place within a digit group. Table~\ref{tab:digit-names} lists the names of
digits in each position.

The preposition \trans{nuh dhihn} (``with negation'') can be used to negate a
numeral. Integers with absolute value less than decimal $12^3$ can be expressed
using just a single digit group. For numbers with magnitude greater or equal to
$12^3$ or with a fractional component, each group of three digits is preceded by
the word \trans{lihnh} and a multiplier. The value of each digit group is its
base value times $12^3$ raised to the power of its multiplier. Note that
negative multipliers express fractional values.

The following extended Backus-Naur form (EBNF) grammar defines the structure of
Najan numerals.

\begin{ebnf}
	\nt{numeral} \is \opt{\t{\naj{dhihn}}} (\nt{group} \alt \plus{\t{\naj{lihnh}} \nt{numeral} \nt{group}})

	\nt{group} \is \nt{triple} \opt{\nt{double}} \opt{\nt{single}}
	\altLine \nt{double} \opt{\nt{single}}
	\altLine \nt{single}
	\altLine \t{\naj{anh}}

	\nt{single} \is \t{\naj{aj}} \alt \t{\naj{az}} \alt \t{\naj{ac}} \alt \t{\naj{ak}} \alt \t{\naj{af}} \alt \t{\naj{ash}} \alt \t{\naj{av}}

	\nt{double} \is \t{\naj{ej}} \alt \t{\naj{ez}} \alt \t{\naj{ec}} \alt \t{\naj{ek}} \alt \t{\naj{ef}} \alt \t{\naj{esh}} \alt \t{\naj{ev}}

	\nt{triple} \is \t{\naj{oj}} \alt \t{\naj{oz}} \alt \t{\naj{oc}} \alt \t{\naj{ok}} \alt \t{\naj{of}} \alt \t{\naj{osh}} \alt \t{\naj{ov}}
\end{ebnf}

\begin{table}
	\caption{Digit names}
	\centering
	\begin{tabular}{llll}
		\toprule
		\multirow{2}{*}{Digit} & \multicolumn{3}{c}{Multiplier}                             \\
		\cmidrule{2-4}
		                       & $\times 1$                     & $\times 8$  & $\times 64$ \\
		\midrule
		0                      & \trans{anh}                    & --          & --          \\
		1                      & \trans{aj}                     & \trans{ej}  & \trans{oj}  \\
		2                      & \trans{az}                     & \trans{ez}  & \trans{oz}  \\
		3                      & \trans{ac}                     & \trans{ec}  & \trans{oc}  \\
		4                      & \trans{ak}                     & \trans{ek}  & \trans{ok}  \\
		5                      & \trans{af}                     & \trans{ef}  & \trans{of}  \\
		6                      & \trans{ash}                    & \trans{esh} & \trans{osh} \\
		7                      & \trans{av}                     & \trans{ev}  & \trans{ov}  \\
		\bottomrule
	\end{tabular}
	\label{tab:digit-names}
\end{table}

\end{document}
