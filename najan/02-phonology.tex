\section{Phonology} \label{sec:phonology}

Tables~\ref{tab:consonants} and \ref{tab:vowels} list the consonants and vowels
of Najan, respectively.

Gemination occurs only between consonants with the same point of articulation.
For instance, the /k/ in \trans{shekga} (``string'') is geminated and not
aspirated since /k/ and /g/ are both velar, but the /k/ in \trans{daktu}
(``enemy'') is aspirated as usual since /t/ is alveolar.

Approximants only occur next to vowels or other approximants; they cannot serve
as the only vowel in a syllable. Within a single word, there can be at most two
consecutive vowels. Including approximants, very long sequences of consonants
are possible, e.g. in \trans{aylmshtwath} (``gambeson'').

Word stress falls on the first syllable of a word, and verbs and nouns usually
receive greater stress than modifiers and particles. However, word stress never
distinguishes words and can be altered for emphasis.

\begin{table*}
	\centering
	\caption{Najan consonants}
	\begin{tabular}{lcccccccc}
		\toprule
		            & Bilabial        & Labio-dental    & Dental            & Alveolar          & Post-alveolar     & Palatal & Velar             \\
		\midrule
		Nasal       & \ipa{m}         &                 &                   & \ipa{n}           &                   &         & \ipa{nh}          \\
		\midrule
		Stop        & \ipa{p} \ipa{b} &                 &                   & \ipa{t} \ipa{d}   &                   &         & \ipa{k} \ipa{g}   \\
		\midrule
		Affricate   &                 &                 &                   & \ipa{tq} \ipa{dq} & \ipa{c} \ipa{j}   &         & \ipa{kq} \ipa{gq} \\
		\midrule
		Fricative   &                 & \ipa{f} \ipa{v} & \ipa{th} \ipa{dh} & \ipa{s} \ipa{z}   & \ipa{sh} \ipa{zh} &         & \ipa{kh} \ipa{gh} \\
		\midrule
		Approximant &                 &                 &                   & \ipa{l}           &                   & \ipa{y} & \ipa{w}           \\
		\midrule
		Trill       &                 &                 &                   & \ipa{xh} \ipa{rh} &                   &         &                   \\
		\bottomrule
	\end{tabular}
	\label{tab:consonants}
\end{table*}

\begin{table}
	\centering
	\caption{Najan vowels}
	\resizebox{0.6\columnwidth}{!}{
		\begin{vowel}
			\putcvowel{i}{1}
			\putcvowel{\ipa{e}}{3}
			\putcvowel{\ipa{a}}{5}
			\putcvowel{\ipa{o}}{7}
			\putcvowel{u}{8}
			\putcvowel{\ipa{ih}}{13}
			\putcvowel{\ipa{uh}}{14}
		\end{vowel}
	}
	\label{tab:vowels}
\end{table}
