\documentclass{article}
\usepackage[english]{babel}
\usepackage{fontspec}
\usepackage{amssymb}
\usepackage[margin=0.7in]{geometry}
\usepackage{expl3}

% Use a font that supports IPA symbols.
\setmainfont{FreeSerif}

% Define Najan font.
\newfontfamily{\najanFont}{[Najan.ttf]}
% Define bold Najan font.
\newfontfamily{\najanFontBold}{[NajanBold.ttf]}

% Convert a transliteration to a version with no digraphs (or n-graphs).
\ExplSyntaxOn
\tl_new:N \l_nodigraphs_tl
\cs_new:Npn \nodigraphs #1 {
	\tl_set:Nn \l_najan_tl {#1}
	\regex_replace_all:nnN {gq} {Q} \l_najan_tl
	\regex_replace_all:nnN {tq} {C} \l_najan_tl
	\regex_replace_all:nnN {dq} {J} \l_najan_tl
	\regex_replace_all:nnN {kq} {q} \l_najan_tl
	\regex_replace_all:nnN {gh} {H} \l_najan_tl
	\regex_replace_all:nnN {xh} {X} \l_najan_tl
	\regex_replace_all:nnN {rh} {R} \l_najan_tl
	\regex_replace_all:nnN {sh} {S} \l_najan_tl
	\regex_replace_all:nnN {zh} {Z} \l_najan_tl
	\regex_replace_all:nnN {th} {T} \l_najan_tl
	\regex_replace_all:nnN {dh} {D} \l_najan_tl
	\regex_replace_all:nnN {nh} {N} \l_najan_tl
	\regex_replace_all:nnN {ih} {I} \l_najan_tl
	\regex_replace_all:nnN {uh} {U} \l_najan_tl
	\regex_replace_all:nnN {kh} {h} \l_najan_tl
	\regex_replace_all:nnN {/ls7/} {¡} \l_najan_tl
	\regex_replace_all:nnN {/ls6/} {¢} \l_najan_tl
	\regex_replace_all:nnN {/ls5/} {£} \l_najan_tl
	\regex_replace_all:nnN {/ls4/} {¤} \l_najan_tl
	\regex_replace_all:nnN {/ls3/} {¥} \l_najan_tl
	\regex_replace_all:nnN {/ls2/} {¦} \l_najan_tl
	\regex_replace_all:nnN {/ls1/} {§} \l_najan_tl
	\regex_replace_all:nnN {/rs1/} {¨} \l_najan_tl
	\regex_replace_all:nnN {/rs2/} {©} \l_najan_tl
	\regex_replace_all:nnN {/rs3/} {ª} \l_najan_tl
	\regex_replace_all:nnN {/rs4/} {«} \l_najan_tl
	\regex_replace_all:nnN {/rs5/} {¬} \l_najan_tl
	\regex_replace_all:nnN {/rs6/} {®} \l_najan_tl
	\regex_replace_all:nnN {/rs7/} {¯} \l_najan_tl
	\tl_use:N \l_najan_tl
}
\ExplSyntaxOff

% Render a transliteration in najan characters.
\NewDocumentCommand{\naj}{m}{{\hyphenrules{nohyphenation}\najanFont\nodigraphs{#1}}}

% Render a transliteration in bold najan characters.
\NewDocumentCommand{\najBold}{m}{{\hyphenrules{nohyphenation}\najanFontBold\nodigraphs{#1}}}

% Render a transliteration in IPA.
\ExplSyntaxOn
\tl_new:N \l_ipa_tl
\cs_new:Npn \ipa #1 {
	\tl_set:Nn \l_najan_tl {#1}
	\regex_replace_all:nnN {gq} {Q} \l_najan_tl
	\regex_replace_all:nnN {tq} {C} \l_najan_tl
	\regex_replace_all:nnN {dq} {J} \l_najan_tl
	\regex_replace_all:nnN {kq} {q} \l_najan_tl
	\regex_replace_all:nnN {gh} {H} \l_najan_tl
	\regex_replace_all:nnN {xh} {X} \l_najan_tl
	\regex_replace_all:nnN {rh} {R} \l_najan_tl
	\regex_replace_all:nnN {sh} {S} \l_najan_tl
	\regex_replace_all:nnN {zh} {Z} \l_najan_tl
	\regex_replace_all:nnN {th} {T} \l_najan_tl
	\regex_replace_all:nnN {dh} {D} \l_najan_tl
	\regex_replace_all:nnN {nh} {N} \l_najan_tl
	\regex_replace_all:nnN {ih} {I} \l_najan_tl
	\regex_replace_all:nnN {uh} {U} \l_najan_tl
	\regex_replace_all:nnN {kh} {h} \l_najan_tl
	\regex_replace_all:nnN {k} {kʰ} \l_najan_tl
	\regex_replace_all:nnN {x} {ɾ̥ʰ} \l_najan_tl
	\regex_replace_all:nnN {r} {ɾ} \l_najan_tl
	\regex_replace_all:nnN {t} {tʰ} \l_najan_tl
	\regex_replace_all:nnN {p} {pʰ} \l_najan_tl
	\regex_replace_all:nnN {q} {k͡xʰ} \l_najan_tl
	\regex_replace_all:nnN {Q} {g͡ɣ} \l_najan_tl
	\regex_replace_all:nnN {C} {t͡sʰ} \l_najan_tl
	\regex_replace_all:nnN {J} {d͡z} \l_najan_tl
	\regex_replace_all:nnN {c} {t͡ʃʰ} \l_najan_tl
	\regex_replace_all:nnN {j} {d͡ʒ} \l_najan_tl
	\regex_replace_all:nnN {h} {x} \l_najan_tl
	\regex_replace_all:nnN {H} {ɣ} \l_najan_tl
	\regex_replace_all:nnN {X} {r̥} \l_najan_tl
	\regex_replace_all:nnN {R} {r} \l_najan_tl
	\regex_replace_all:nnN {S} {ʃ} \l_najan_tl
	\regex_replace_all:nnN {Z} {ʒ} \l_najan_tl
	\regex_replace_all:nnN {T} {θ} \l_najan_tl
	\regex_replace_all:nnN {D} {ð} \l_najan_tl
	\regex_replace_all:nnN {N} {ŋ} \l_najan_tl
	\regex_replace_all:nnN {y} {j} \l_najan_tl
	\regex_replace_all:nnN {U} {ʊ} \l_najan_tl
	\regex_replace_all:nnN {a} {ɑ} \l_najan_tl
	\regex_replace_all:nnN {e} {ɛ} \l_najan_tl
	\regex_replace_all:nnN {I} {ɪ} \l_najan_tl
	\regex_replace_all:nnN {i} {i} \l_najan_tl
	\regex_replace_all:nnN {u} {u} \l_najan_tl
	\tl_use:N \l_najan_tl
}
\ExplSyntaxOff

% Writes the given transliteration in Najan script followed by the italicized transliteration in parentheses.
\NewDocumentCommand{\trans}{m}{\naj{#1} ⟨#1⟩}

% Writes the given transliteration in Najan script followed by the italicized transliteration in parentheses on a new line.
\NewDocumentCommand{\transLine}{m}{\naj{#1} \\ ⟨#1⟩}

\usepackage[title]{appendix}
\usepackage{booktabs}
\usepackage{float}
\usepackage
    [ colorlinks
    , linkcolor=.
    , urlcolor=blue
    ]{hyperref}
\usepackage[usenames, dvipsnames]{xcolor} % Must include before mdframed.
\usepackage[framemethod=tikz]{mdframed}
\usepackage{multirow}
\usepackage[super]{nth}
\usepackage[parfill]{parskip}
\usepackage{tcolorbox}
\usepackage{todonotes}
\usepackage{vowel}

\NewDocumentCommand{\grapheme}{m}{\huge \naj{#1}}
\NewDocumentCommand{\transIPA}{m}{⟨#1⟩ [\ipa{#1}]}

\definecolor{NonterminalColor}{RGB}{0, 32, 96}
\definecolor{TerminalColor}{RGB}{60, 128, 49}

\NewDocumentEnvironment{ebnf}{}{
	% Formatting settings.
	\setlength{\parindent}{0em}
	\setlength{\parskip}{0.5\baselineskip}
	% Separates LHS and RHS.
	\newcommand{\is}{\ $\rightarrow$ \ }
	% Alternative.
	\newcommand{\alt}{|\ }
	\newcommand{\altLine}{\\ \phantom{\hspace{2em}} |\ \ \ }
	% Puts the RHS on a new line without an alternative symbol.
	\newcommand{\newLine}{\\ \phantom{\hspace{2em}}}
	% Optional group.
	\NewDocumentCommand{\opt}{m}{##1?}
	% Zero or more copies.
	\RenewDocumentCommand{\star}{m}{##1\textsuperscript{*}}
	% One or more copies.
	\NewDocumentCommand{\plus}{m}{##1\textsuperscript{+}}
	% N copies.
	\NewDocumentCommand{\n}{mm}{##1\textsuperscript{##2}}
	% Nonterminal symbol.
	\NewDocumentCommand{\nt}{m}{⟨\textcolor{NonterminalColor}{##1}⟩}
	% Terminal symbol in Najan.
	\RenewDocumentCommand{\t}{m}{``\textcolor{TerminalColor}{##1}''}

	% Box the grammar.
	\begin{mdframed}
		[ hidealllines=true
		, backgroundcolor=gray!10
		, innerleftmargin=3pt
		, innerrightmargin=3pt
		, leftmargin=-3pt
		, rightmargin=-3pt
		]
}{
	\end{mdframed}
}

% Box to hold an example translation from English to Najan.
\newcounter{example}
\stepcounter{example}
\NewDocumentCommand{\example}{mmm}{
	\vspace{1em}
	\begin{tcolorbox}
		[ title=Example \theexample: #1
		, halign=center
		, halign lower=center
		]
		#2 \tcblower #3
	\end{tcolorbox}
	\vspace{0.5em}
	\stepcounter{example}
}

% Down arrow and new line, for use in translation examples.
\NewDocumentCommand{\down}{}{$\Downarrow$ \\}

\title{\naj{kuhv naj vihrh} \\ The Najan Language}
\author{Jonathan Sharman}

\begin{document}

\listoftodos

\maketitle

\tableofcontents

\twocolumn

\section{Orthography} \label{sec:orthography}

Najan orthography is perfectly phonemic, i.e. there is a bijection between
graphemes and phonemes. Table~\ref{tab:alphabet} lists the thirty-seven letters
of the Najan alphabet, along with the transliteration and IPA transcription for
each letter.

Sentences are written left-to-right, then top-to-bottom. The only punctuation
mark in written Najan is the sentence separator, ``\naj{.}'', which is used to
separate independent sentences and/or interjections.

\begin{table}
	\caption{The Najan alphabet}
	\centering
	\resizebox{\columnwidth}{!}{\begin{tabular}{cccccc}
			\toprule
			\grapheme{k}  & \grapheme{g}  & \grapheme{t}  & \grapheme{d}  & \grapheme{p} & \grapheme{b} \\
			\transIPA{k}  & \transIPA{g}  & \transIPA{t}  & \transIPA{d}  & \transIPA{p} & \transIPA{b} \\
			\midrule

			\grapheme{kq} & \grapheme{gq} & \grapheme{tq} & \grapheme{dq} & \grapheme{c} & \grapheme{j} \\
			\transIPA{kq} & \transIPA{gq} & \transIPA{tq} & \transIPA{dq} & \transIPA{c} & \transIPA{j} \\
			\midrule

			\grapheme{kh} & \grapheme{gh} & \grapheme{x}  & \grapheme{r}  & \grapheme{s} & \grapheme{z} \\
			\transIPA{kh} & \transIPA{gh} & \transIPA{x}  & \transIPA{r}  & \transIPA{s} & \transIPA{z} \\
			\midrule

			\grapheme{sh} & \grapheme{zh} & \grapheme{th} & \grapheme{dh} & \grapheme{f} & \grapheme{v} \\
			\transIPA{sh} & \transIPA{zh} & \transIPA{th} & \transIPA{dh} & \transIPA{f} & \transIPA{v} \\
			\midrule

			\grapheme{nh} & \grapheme{n}  & \grapheme{m}  & \grapheme{y}  & \grapheme{l} & \grapheme{w} \\
			\transIPA{nh} & \transIPA{n}  & \transIPA{m}  & \transIPA{y}  & \transIPA{l} & \transIPA{w} \\
			\midrule
		\end{tabular}}
	\resizebox{\columnwidth}{!}{\begin{tabular}{ccccccc}
			\grapheme{uh} & \grapheme{a} & \grapheme{e} & \grapheme{ih} & \grapheme{i} & \grapheme{u} & \grapheme{o} \\
			\transIPA{uh} & \transIPA{a} & \transIPA{e} & \transIPA{ih} & \transIPA{i} & \transIPA{u} & \transIPA{o} \\
			\bottomrule
		\end{tabular}}
	\label{tab:alphabet}
\end{table}

\subsection{Numbers}

\todo{Switch to dozenal.}

Najan uses an octal positional numeral system and thus has eight numeric glyphs
(0-7): \naj{0}, \naj{1}, \naj{2}, \naj{3}, \naj{4}, \naj{5}, \naj{6}, and
\naj{7}. Numerals are typically written using these glyphs (from most to least
significant digit) rather than spelling them out as they are pronounced (as
described in \S\ref{sec:morphology}).

\section{Morphology} \label{sec:morphology}

Najan is analytic and purely isolating (having just one morpheme per word).

\section{Syntax} \label{sec:syntax}

The formal grammar of the language is defined by the following extended
Backus-Naur form (EBNF) grammar. This is a deterministic context-free grammar,
making it amenable to automatic parsing.

\begin{ebnf}
	\nt{clause} \is \opt{\nt{mood particle}} \nt{verb phrase (VP)}
	\\
	\nt{VP} \is \opt{\nt{aspect particle}} \nt{verb}
	\altLine \nt{prep.} \nt{NP} \nt{VP}
	\altLine \nt{conj.} \nt{VP} \nt{VP}
	\\
	\nt{noun phrase (NP)} \is \nt{quant. NP}
	\altLine \nt{unquant. NP}
	\altLine \nt{clause}
	\altLine \nt{quotation}
	\altLine \nt{conj.} \nt{NP} \nt{NP}
	\\
	\nt{quant. NP} \is \nt{det.} \plus{\nt{pronoun}} \nt{unquant. NP}
	\\
	\nt{unquant. NP} \is \nt{noun} \alt \nt{pronoun}
	\altLine \nt{prep.} \nt{NP} \nt{unquant. NP}
	\\
	\nt{quotation} \is \t{\trans{ca}} \star{\nt{quoted word}} \t{\trans{ca}}
	\\
	\nt{quoted word} \is \nt{word} - \t{\trans{ca}} - \t{\trans{nosh}} - \t{\trans{tqa}}
	\altLine \t{\trans{tqa}} \t{\trans{ca}}
	\altLine \t{\trans{tqa}} \t{\trans{nosh}}
	\altLine \t{\trans{tqa}} \t{\trans{tqa}}
\end{ebnf}

\todo{Example}

\paragraph{Quotation} A quotation begins and ends with the particle \trans{ca},
acting as a noun signifying the literal contained words themselves. The escape
particle \trans{tqa} is used within a quotation just before \trans{ca},
\trans{nosh}, or \trans{tqa} to indicate that the following particle should be
interpreted as part of the quotation, not as a particle within the sentence
containing the quotation.

\example{Nested quotation using particles}{
	You say ``I say `no'''.
}{
	\transLine{ka ca ka ca esh tqa sa tihz ko sa tihz to}
}

\section{Binding and Quantification} \label{sec:binding}

\paragraph{Quantified Terms} A quantified term binds one or more pronouns to a
complement, according to the semantics of a determiner, and signifies the bound
pronouns in-place. Generally, such pronouns remain bound until they are rebound
to a different complement. However, any pronouns bound using the universal
determiner \trans{u}---or in any subsequent quantified term within the same
clause---become unbound at the end of the current clause.

\todo{ Explain the semantics of multiple pronouns. (I think it'll be equivalent
	to repeating the containing phrase.) }

\todo{ Examples illustrating why it works this way. "Every man X has a horse Y;
	Y is a good horse" doesn't make sense. "A man X has a horse Y; Y is a good
	horse" does }

\paragraph{Quotations} Quotations cannot affect the bindings of surrounding
pronouns. Any assignable pronouns contained in a quotation may refer to
completely different antecedents that in the surrounding context or even to
nothing at all, e.g. when using a quotation to talk about the pronouns
themselves.

\subsection{Pronouns} \label{sec:pronouns}

\paragraph{Assignable} Assignable pronouns are those which can be bound
explicitly using determiners. There are four assignable pronouns: \trans{zho},
\trans{co}, \trans{rho}, and \trans{so}. Once all available pronouns have been
assigned, subsequent assignments must overwrite an existing assignment.

\paragraph{Personal} Najan has four personal pronouns, which are listed in
Table~\ref{tab:personal-pronouns}.

\begin{table}
	\caption{Personal pronouns}
	\centering
	\begin{tabular}{lll}
		\toprule
		               & Singular   & Plural       \\
		\midrule
		\nth{1}-person & \trans{ko} & \trans{kxho} \\
		\nth{2}-person & \trans{to} & \trans{txho} \\
		\bottomrule
	\end{tabular}
	\label{tab:personal-pronouns}
\end{table}

Note that the first-person plural (\trans{kxho}) is used only in speech or
writing considered to have multiple speakers or authors, to refer to the group a
spokesperson represents or to the coauthors of a jointly written document. In
other words, the first-person plural in Najan does not merely indicate
association with the speaker; it indicates multiple speakers. Unlike in English,
the first-person plural should not be used to refer to one's group when speaking
as an individual or to refer to oneself and the addressee together.

The choice between second-person singular or plural is more intuitive, from an
English-language perspective. If there is a single primary addressee, the
singular is used. If there are multiple addressees, the plural is used.

\todo{I could probably get rid of the plural personal pronouns.}

\paragraph{Interrogative} The interrogative pronoun \trans{vo} is used in place
of unknown terms. It is also essential to the formation of interrogative
sentences, where it indicates any missing information that the asker wishes to
know.

\subsection{Determiners} \label{sec:determiners}

\paragraph{Universal} The universal determiner \trans{u} is used to express that
its immediately enclosing clause holds for each instance of its complement,
functioning similarly to English ``each'', ``every'', or ``all''. When it binds
$k$ pronouns in a single quantified noun phrase, the clause applies to every
$k$-permutation of the elements. As noted above, the universal determiner is
unique in that it only binds pronouns for the duration of the containing clause.

\example{Universal determiner}
{Everyone hugged each other.}
{\transLine{u zho co mil the buda va zho co}}

\paragraph{Existential} The existential determiner \trans{a} binds pronouns to
individual instances of a complement, similar to many uses of English
``a''/``an''. When it binds multiple pronouns, each signifies a unique instance.
Pronouns bound by separate existential determiners applied to the same
complement may or may not share a referent.

\example{Existential determiner}
{I have a dog.}
{\transLine{ya lut ko a zho nhiruh}}

\todo{ Rework this old example, talking about shifting modifiers around instead.
	Also include an example using a specialized ``for'' modifier, for when the
	pronouns need to be defined together but used separately? }

\example{Using forward assignment to change semantics}{
	Every person has an ancestor. \\
	$\Downarrow$ \\
	For each person, there's a person that's an ancestor to the first. \\
}{
	\transLine{u zho mil ya nhinh a co mil OF runat zho} \\
	$\Downarrow$ \\
	\transLine{u zho mil a co mil ya nhinh co OF runat zho} \\
}

\paragraph{Definite} When the set of entities an unquantified term refers to is
a singleton, i.e. when it has only one instance, the definitive determiner
\trans{i} can be used to refer to that single instance, functioning much the
same as the English determiner ``the'' in many cases.

\example{Definite determiner}{
	A dog wagged its tail. \\
	$\Downarrow$ \\
	Before now, a dog wagged the tail belonging to it.
}{
	\transLine{the va threre a zho nhiruh i co sex zho shuht}
}

\todo{Do I really want a definite determiner?}

\paragraph{Demonstrative} Demonstrative determiners function just like the
definite determiner except that the complement need not be a singleton. Instead,
the particular instance of the complement is understood by context, e.g. by
physical gesturing or by common understanding. Najan has proximal
(\trans{ihla}), medial (\trans{aya}), and distal (\trans{owa}) demonstrative
determiners, based on the distance of the referent from the speaker in space and
time. These correspond roughly to English ``this'', ``that'', and ``yonder'',
respectively. The concept of distance with regard to demonstrative determiners
can be more abstract than just spatiotemporal distance to a physical object;
e.g. it could be a measure of the relevance of the complement to the present
conversation.

\paragraph{Indefinite} The indefinite determiner \trans{na} acts like the
English word ``any'', meaning a single instance of the complement where the
particular instance does not matter. As with the existential determiner, it can
bind multiple pronouns at once, in which case each referent is unique. Note that
many English sentences using ``a''/``an'' are best translated using \trans{na}
rather than \trans{a}.

\example{Indefinite determiner}{
	I want a book.
}{
	\transLine{ka na zho suaysh zihm ko}
	\\
	acting-on any pron. book want I
}

\example{Indefinite determiner binding two pronouns}{
	I want two books.
}{
	\transLine{wa na zho co suaysh ka sem zho co zihm ko}
	\\
	forward-assignment any pron.-1 pron.-2 book acting-on and pron.-1 pron.-2 want I
}

\todo{ This would probably be better translated as "I want any books numbering
	two" unless the specific two books are going to be referenced individually
	later. Might want to come up with a better example or add a sentence using
	the books. }

\subsection{Cancellation} \label{sec:cancellation}

The cancellation particle \trans{nosh} can be used to cancel the entire
preceding independent sentence, even if it is only partially uttered. The
conversation then proceeds as if the utterance had never occurred, including
reverting the context to its previous state. Speakers and listeners alike can
use \trans{nosh} at any time in a conversation. Cancellation is useful when the
speaker makes a mistake (like saying ``I mean'') or when the listener does not
understand something (like saying ``What?'').

Note that for the sake of simplicity and clarity, the formal grammar
(\S\ref{sec:syntax}) does not handle cancellation. It is relatively
straightforward for a parser to detect cancellation (e.g., by checking for an
unescaped cancellation particle during tokenization), but expressing it formally
overcomplicates the grammar.

\section{Conjunctions} \label{sec:conjunctions}

In Najan, conjunctions are restricted to logical connectives, i.e. functions of
Boolean values. Many words considered conjunctions in other languages, such as
English ``because'', are instead classified as modifiers in Najan. Conjunctions
can connect sentences, predicates, or terms and can be unary, binary, or
variadic. In the case of variadic conjunctions, \trans{sa} marks the end of the
list of arguments. Table~\ref{tab:conjunctions} lists all conjunctions in Najan.

\begin{table}
	\caption{Conjunctions}
	\centering
	\begin{tabular*}{\columnwidth}{@{\extracolsep{\fill}} llll}
		\toprule
		Function & Sentence & Predicate & Term \\
		\midrule
		$\lnot$ ({\sc not}) & \trans{nunh} & -- & -- \\
		$\land$ ({\sc and}) & \trans{senh} & \trans{sen} & \trans{sem} \\
		$\lor$ ({\sc or}) & \trans{shenh} & \trans{shen} & \trans{shem} \\
		$\oplus$ ({\sc xor}) & \trans{cenh} & \trans{cen} & \trans{cem} \\
		$\rightarrow$ ({\sc implies}) & \trans{tqanh} & \trans{tqan} & \trans{tqam} \\
		\bottomrule
	\end{tabular*}
	\label{tab:conjunctions}
\end{table}

\todo{ Do I need three versions of each conjunction, or can some of them be
	shared and distinguished by syntax? }

Note that biconditional statements (``if and only if'') are equivalent to
	{\sc xnor}, which can be expressed by combining {\sc not} and {\sc xor} conjunctions.

Note that \trans{tqanh} only denotes material implication; it does not imply a
causal relationship.

\subsection{Conjunction Rewriting Rules} \label{sec:conj-rewriting}

Sentential conjunctions are the fundamental and most general form of
conjunction. Semantically, any conjunction of terms or predicates can be
expressed using a sentential conjunction instead, though usually less concisely.
As with the conversion from quantified terms to forward assignments
(\S\ref{sec:binding}), this is a mechanical transformation. A sentence can be
converted to only sentential conjunctions as follows:

\begin{enumerate}
	\item Factor out in-place quantified terms to forward assignments, as
	      described in \S\ref{sec:binding}. Now consider the subsentence after
	      all forward assignments.
	\item Scanning left to right, if there is a term or predicate conjunction,
	      replace the containing sentence with the corresponding sentential
	      conjunction with copies of the sentence, each holding one of the conjunction
	      arguments in place of the original conjunction.
	\item If you performed any substitutions, repeat steps 2-3 for each new
	      subsentence.
\end{enumerate}

\example{Conversion from a term conjunction to a sentence conjunction}{
	I have a cat and a dog.
}{
	\transLine{lut ko sem a zho kanaz a co nhiruh} \\
	$\Downarrow$ \\
	\transLine{a zho kanaz a co nhiruh lut ko sem zho co} \\
	$\Downarrow$ \\
	\transLine{a zho kanaz a co nhiruh senh lut ko zho lut ko co}. \\
}

Special care should be taken when translating conjunctions from English to
Najan. A simple literal translation often provides the wrong semantics. Consider
the following translation attempt:

\example{Incorrect translation of English conjunction}
{I want an apple or an orange.}
{WRONG: \trans{zihm ko cem na zho duhxihn na co rhabuhm}}

The translated sentence actually means ``Either I want an apple, or I want an
orange.'' A more faithful translation of the original sentence would be as
follows:

\example{Correct translation of conjunction using union modifier}
{I want an apple or an orange.}
{\transLine{zihm ko na zho zaw duhxihn rhabuhm}}

This literally means that the speaker wants any single item that is either an
apple or an orange.

\section{Verbs} \label{sec:verbs}

\subsection{Mood} \label{sec:mood}

Clauses default to the subjunctive mood but can be changed using a particle
before the clause.

\paragraph{Subjunctive} Used for talking about clauses in the abstract,
subjunctive clauses don't signify anything as independent clauses; they are
always used as a dependent clause (e.g. in a hypothetical) or other noun phrase,
serving a role similar to verbal nouns in English. The shortest possible
subjunctive clauses - just the particle and a verb, with no modifiers - can
function exactly like an infinitive or gerund in English.

\paragraph{Indicative} Signifies that the speaker believes the clause is true.

\paragraph{Deontic} Expresses wishes and commands. Najan does not have a
separate imperative mood to express requests or orders; a deontic clause with a
(perhaps implicit) second-person agent serves this purpose.

\paragraph{Polar Interrogative} Express yes-no questions. The particle
\trans{kya} could be translated as ``it is being asked whether [clause]''. The
expectation is that the listener will answer by affirming the question
(\trans{azh}), negating it (\trans{esh}), or objecting to its underlying
premises or implications (\trans{ith}).

\todo{Update this example.}

\example{Polar interrogative sentence}{
	\begin{tabular}{rl}
		\textbf{Alice} & Have you stopped beating your spouse? \\
		\textbf{Bob}   & To the contrary, I never beat her.    \\
	\end{tabular}
}{
	\begin{tabular}{rl}
		% Duplicating text here instead of using a macro to make the
		% table rows and columns work.
		\textbf{Alice} & \naj{kya kxesh goruhp to i zho nih enihd to} \\
		               & ⟨kya kxesh goruhp to i zho nih enihd to      \\
		\textbf{Bob}   & \naj{ith. nu goruhp ko zho}                  \\
		               & ⟨ith. nu goruhp ko zho⟩                      \\
	\end{tabular}
}

Multiple-choice questions can be constructed using the polar interrogative mood
and a logical disjunction, which due to the rewriting rules defined in
\S~\ref{sec:conj-rewriting} has the same semantics as asking multiple yes/no
questions.

\paragraph{Open Interrogative} Used for open-ended questions. An open
interrogative sentence must include at least one occurrence of the interrogative
pronoun, \trans{vo}. The utterance of an open interrogative clause invites the
listener to supply an answer for each occurrence of \trans{vo}. As with polar
questions, \trans{ihth} can be an appropriate response if the question is based
on false premises and therefore cannot be meaningfully answered.

\example{Open interrogative sentence}
{What do you want?}
{\transLine{kwa ya kih FOR vo vihm to}}

\example{Open interrogative sentence with discrete choices}
{Do you want soup or salad?}
{\transLine{kwa ya vihm nhinh zaw thloth pewsh vo to}}

\paragraph{Confirmative Interrogative} This mood is similar to the polar
interrogative except that the speaker already believes the proposition to be
true, at least tentatively. Use of this mood presents the addressee a chance to
contradict the claim if it is false. It may also be used rhetorically when the
speaker is already certain the claim is true. Sentences of this mood are like
tag questions in English (``right?'') and other languages. \trans{odh} and
\trans{ihth} are typical responses to confirmative questions.

\todo{Make these particles more dissimilar and interesting.}

\begin{table}
	\caption{Mood particles}
	\centering
	\begin{tabular}{ll}
		\toprule
		Mood                       & Particle     \\
		\midrule
		Subjunctive                & --           \\
		Indicative                 & \trans{zhe}  \\
		Deontic                    & \trans{ksha} \\
		Polar Interrogative        & \trans{kya}  \\
		Open Interrogative         & \trans{kwa}  \\
		Confirmative Interrogative & \trans{kla}  \\
		\bottomrule
	\end{tabular}
	\label{tab:mood-particles}
\end{table}

\subsection{Aspect} \label{sec:aspect}

Verbs default to the continuous aspect but can be changed using a particle
immediately preceding the verb. (Note that the continuous aspect applies to both
dynamic and stative verbs. Najan does not distinguish between progressive and
continuous aspects.)

\begin{table}
	\caption{Aspect particles}
	\centering
	\begin{tabular}{ll}
		\toprule
		Particle     & Aspect
		\midrule
		--           & Continuous \\
		\trans{the}  & Perfective \\
		\trans{fxe}  & Habitual   \\
		\trans{khon} & Gnomic     \\
		\bottomrule
	\end{tabular}
	\label{tab:aspect-particles}
\end{table}

\subsection{Arguments} \label{sec:arguments}

All arguments to the verb are expressed via optional prepositional phrases
modifying the verb (\S\ref{sec:modifiers}). Since prepositional phrases can
occur in any order, Najan has very free word order.

In English, the subject does not vary morphosyntactically when paired with a
dynamic vs. a stative verb. In Najan however, dynamic and stative clauses
require different sets of prepositional phrases to express their arguments.

Common prepositions for expressing dynamic arguments are listed in
Table~\ref{tab:dynamic-argument-prepositions}.

\todo{Use descriptive paragraphs instead of a table.}

\begin{table}
	\caption{Dynamic argument prepositions}
	\centering
	\begin{tabular}{ll}
		\toprule
		Relation          & Modifier      \\
		\midrule
		Agent             & \trans{shi}   \\
		Involuntary cause & \trans{slo}   \\
		Patient           & \trans{ruh}   \\
		Instrument        & \trans{lathu} \\
		Recipient         & \trans{dhu}   \\
		\bottomrule
	\end{tabular}
	\label{tab:dynamic-argument-prepositions}
\end{table}

\todo{Examples. "I destroy the sand castle" vs. "The sea destroys the sand castle"}

The prepositions for stative arguments are more abstract. The subject marker
\trans{nihm} indicates the primary noun phrase to which the predicate applies.
The object marker \trans{gha} provides additional context. The exact semantics
of these arguments varies by stative verb.

\todo{Example: this equals that: this and that are both subjects}

\todo{Example: this is greater than that: this is the subject, that the object}

Note that because all arguments to the verb are indicated using modifiers, it's
possible for a verb to have multiple arguments of the same type - for instance,
multiple agents - without using grouping phrases.

\todo{Format and translate examples below.}

\trans{shi ko shi to <sing>} implies singing together.

\trans{shi ko ruh <ball> ruh <glove> <throw>} implies I threw the ball
and glove as part of the same action.

\subsection{Tense} \label{sec:tense}

Like arguments to the verb, verb tense is expressed by modifying the verb with a
prepositional phrase. If no such modifiers are present, then the time frame must
be inferred from context. Table~\ref{tab:temporal-modifiers} lists all the
modifiers that affect the tense of a verb phrase, and
Table~\ref{tab:temporal-terms} provides a (nonexhaustive) list of terms commonly
used with temporal modifiers.

\begin{table}
	\caption{Temporal modifiers}
	\centering
	\begin{tabular}{ll}
		\toprule
		Modifier     & Meaning                     \\
		\midrule
		\trans{ve}   & at/on [time point]          \\
		\trans{le}   & for [duration]              \\
		\trans{de}   & during/in [time period]     \\
		\trans{the}  & before [time point]         \\
		\trans{thih} & shortly before [time point] \\
		\trans{thu}  & long before [time point]    \\
		\trans{she}  & after [time point]          \\
		\trans{shih} & shortly after [time point]  \\
		\trans{shu}  & long after [time point]     \\
		\trans{nhe}  & from [time point]           \\
		\trans{ne}   & to [time point]             \\
		             & until [event]               \\
		             & while [event]               \\
		\bottomrule
	\end{tabular}
	\label{tab:temporal-modifiers}
\end{table}

\todo{That looks like inflection to me.}

\begin{table}
	\caption{Common temporal terms}
	\centering
	\begin{tabular}{ll}
		\toprule
		Term            & Meaning                   \\
		\midrule
		\trans{va}      & the present/now           \\
		\trans{tha}     & the past                  \\
		\trans{sha}     & the future                \\
		\trans{gha}     & moment/instant            \\
		\trans{kha}     & time span                 \\
		\trans{thenish} & eternity/forever/all time \\
		\bottomrule
	\end{tabular}
	\label{tab:temporal-terms}
\end{table}

\subsection{Interjections} \label{sec:interjections}

Some avalent verbs function as interjections. Four common interjections are
\trans{azh} (``yes''), \trans{esh} (``no''), \trans{odh} (``indeed'',
``right''), and \trans{ihth} (``to the contrary'').

\section{Modifiers} \label{sec:modifiers}

\todo{ This entire section below this line may be outdated and in need of
	reworking/redistribution. }

Modifiers change the meaning of sentences, predicates, or terms. They serve the
roles that prepositions, adjectives, adverbs, and certain coordinating
conjunctions serve in English. A modifier precedes any additional arguments
(which are terms), depending on the valency of the modifier, which precede the
modified object.

\subsection{Sentence Modifiers}

\begin{table}
	\caption{Sentence modifiers}
	\centering
	\begin{tabular}{ll}
		\toprule
		Modifier     & Meaning                                  \\
		\midrule
		\trans{bihn} & because [reason]                         \\
		\trans{vihr} & so/thus/therefore [consequence]          \\
		             & with respect to/regarding/concerning ... \\
		\bottomrule
	\end{tabular}
	\label{tab:sentence-modifiers}
\end{table}

\subsection{Predicate Modifiers}

\begin{table}
	\caption{Spatial modifiers}
	\centering
	\begin{tabular}{ll}
		\toprule
		Modifier & Meaning                        \\
		\midrule
		\naj{}   & at [position]                  \\
		\naj{}   & away from [position/direction] \\
		\naj{}   & to [position/direction]        \\
		\naj{}   & towards [position/direction]   \\

		\bottomrule
	\end{tabular}
	\label{tab:spatial-modifiers}
\end{table}

\subsection{Term Modifiers}

\begin{table}
	\caption{Abstract modifiers}
	\centering
	\begin{tabular}{ll}
		\toprule
		Modifier    & Meaning                    \\
		\midrule
		\trans{nih} & of/related to [term]       \\
		\trans{sex} & component/member of [term] \\
		\trans{sih} & such that [sentence]       \\
		\trans{kih} & with/having [property]     \\
		\bottomrule
	\end{tabular}
	\label{tab:abstract-modifiers}
\end{table}

\begin{table}
	\caption{Mathematical modifiers}
	\centering
	\begin{tabular}{ll}
		\toprule
		Modifier      & Meaning                                    \\
		\midrule
		\trans{nhinh} & belonging to/element of [set]              \\
		\trans{vekh}  & subset of [set]                            \\
		\trans{vekq}  & proper subset of [set]                     \\
		\trans{ghaf}  & superset of [set]                          \\
		\trans{gqaf}  & proper superset of [set]                   \\
		\trans{zaw}   & union with [set]                           \\
		\trans{tew}   & intersection with [set]                    \\
		\trans{khep}  & except/set-minus [set]                     \\
		\trans{shi}   & equivalent to/identical to [term]          \\
		\trans{thik}  & equal to [term]                            \\
		\trans{muhl}  & not equal to ...                           \\
		\trans{thuhm} & less than or equal to ...                  \\
		\trans{thuhl} & less than ...                              \\
		\trans{shuhm} & greater than or equal to ...               \\
		\trans{shuhl} & greater than ...                           \\
		\trans{ray}   & preceded by [cardinal number] predecessors \\
		\bottomrule
	\end{tabular}
	\label{tab:mathematical-modifiers}
\end{table}

\todo{Lots of inflection here as well.}

\section{Numbers} \label{sec:numbers}

Numerals always function grammatically as nouns and signify rational numbers.

\paragraph{Ordinals} Ordinality (\trans{ray}) in Najan is zero-based (unlike in
English, where ordinals are one-based).

\example{Ordinal number}{
	the first person \\
	$\Downarrow$ \\
	the person with ordinality equal to zero
}{
	\trans{i nuh thik anh ray mihl}
}

\todo{ This section below this line may be out of date. Should have just twelve
	easily distinguishable digits, combined using modifiers. }

Najan uses a dozenal positional numeral system. Digits are grouped into threes,
with 1,728 possible values per digit group. Placeholder zeros are not
pronounced, and the eleven non-zero digits have unique names depending on their
place within a digit group. Table~\ref{tab:digit-names} lists the names of
digits in each position.

The preposition \trans{nuh dhihn} (``with negation'') can be used to negate a
numeral. Integers with absolute value less than decimal $12^3$ can be expressed
using just a single digit group. For numbers with magnitude greater or equal to
$12^3$ or with a fractional component, each group of three digits is preceded by
the word \trans{lihnh} and a multiplier. The value of each digit group is its
base value times $12^3$ raised to the power of its multiplier. Note that
negative multipliers express fractional values.

The following extended Backus-Naur form (EBNF) grammar defines the structure of
Najan numerals.

\begin{ebnf}
	\nt{numeral} \is \opt{\t{\naj{dhihn}}} (\nt{group} \alt \plus{\t{\naj{lihnh}} \nt{numeral} \nt{group}})

	\nt{group} \is \nt{triple} \opt{\nt{double}} \opt{\nt{single}}
	\altLine \nt{double} \opt{\nt{single}}
	\altLine \nt{single}
	\altLine \t{\naj{anh}}

	\nt{single} \is \t{\naj{aj}} \alt \t{\naj{az}} \alt \t{\naj{ac}} \alt \t{\naj{ak}} \alt \t{\naj{af}} \alt \t{\naj{ash}} \alt \t{\naj{av}}

	\nt{double} \is \t{\naj{ej}} \alt \t{\naj{ez}} \alt \t{\naj{ec}} \alt \t{\naj{ek}} \alt \t{\naj{ef}} \alt \t{\naj{esh}} \alt \t{\naj{ev}}

	\nt{triple} \is \t{\naj{oj}} \alt \t{\naj{oz}} \alt \t{\naj{oc}} \alt \t{\naj{ok}} \alt \t{\naj{of}} \alt \t{\naj{osh}} \alt \t{\naj{ov}}
\end{ebnf}

\begin{table}
	\caption{Digit names}
	\centering
	\begin{tabular}{llll}
		\toprule
		\multirow{2}{*}{Digit} & \multicolumn{3}{c}{Multiplier}                             \\
		\cmidrule{2-4}
		                       & $\times 1$                     & $\times 8$  & $\times 64$ \\
		\midrule
		0                      & \trans{anh}                    & --          & --          \\
		1                      & \trans{aj}                     & \trans{ej}  & \trans{oj}  \\
		2                      & \trans{az}                     & \trans{ez}  & \trans{oz}  \\
		3                      & \trans{ac}                     & \trans{ec}  & \trans{oc}  \\
		4                      & \trans{ak}                     & \trans{ek}  & \trans{ok}  \\
		5                      & \trans{af}                     & \trans{ef}  & \trans{of}  \\
		6                      & \trans{ash}                    & \trans{esh} & \trans{osh} \\
		7                      & \trans{av}                     & \trans{ev}  & \trans{ov}  \\
		\bottomrule
	\end{tabular}
	\label{tab:digit-names}
\end{table}

\begin{appendices}
	\section{Example Translations}
\end{appendices}

\end{document}
